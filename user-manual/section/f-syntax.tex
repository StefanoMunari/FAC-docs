\section{F Syntax}
The syntax is pretty similar to the C one except for the fract type that
is represented by writing \verb![numerator|denominator]!.
The operation that can be done on a value of type fract are described in Table
\ref{table:f:fract:operators} whereas the comparison on this type are reported
in Table \ref{table:f:fract:comparison:operators}.
The operators for the boolean type are listed in table 
\ref{table:f:bool:operators}.





%	&	\verb|+ -|	& Unary plus and minus & Right-to-left \\
%	& \verb|* /| & fract multiplication and division & Left-to-right \\ 
\begin{table}[h]
\centering
\begin{tabular}{|c|c|c|}
\hline
\textbf{Operator} & \textbf{Description} & \textbf{Associativity} \\ 
\hline
\verb|+ -| & Unary plus and minus 	& Right-to-left	\\
\verb|* /| & fract multiplication and division & Left-to-right \\ 
\verb|+ -| & fract sum and difference & Left-to-right \\
\hline
\end{tabular}
\caption{F operators that can be applied to fract values and return results
of the same type.}
\label{table:f:fract:operators}


\end{table}

\begin{table}[h]
\centering
\begin{tabular}{|c|c|c|}
\hline
\textbf{Operator} & \textbf{Description} & \textbf{Associativity} \\ 
\hline
\verb|<|	& less than	& Right-to-left	\\
\verb|<=|	& less than or equal to	& Left-to-right \\ 
\verb|==|	& equality & Left-to-right \\
\verb|<>|	& disequality & Left-to-right \\
\verb|>|	& bigger than & Left-to-right\\
\verb|>=|	& bigger than or equal to & Left-to-right \\
\hline
\end{tabular}
\caption{Fract comparison operators that return boolean values.}
\label{table:f:fract:comparison:operators}
\end{table}


\begin{table}[h]
\centering

\begin{tabular}{|c|l|l|l|l|}
\hline
\textbf{Operator} & \textbf{Description} & \textbf{Associativity} \\ 
\hline
\verb|!|	& Logial negation	& Right-to-left	\\
\verb|<->| & logical biimplication (boolean equality) & Left-to-right \\ 
\verb|XOR| & logical xor (boolean disequality) & Left-to-right \\ 
\verb|->|  & logical implication & Left-to-right \\ 
\verb!||!  & logical or & Left-to-right\\ 
\verb|&&|	& logical and & Left-to-right\\ 
\hline
\end{tabular}
\caption{F operation that can be applied on boolean operator}
\label{table:f:bool:operators}
\end{table}


