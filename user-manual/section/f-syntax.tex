\section{F Syntax}
The syntax is pretty similar to the C one except for the fract type that
is represented by writing \verb![numerator|denominator]!.
The available operator for fract type are described in Table
\ref{table:f:fract:operators} whereas the comparison on this type are reported
in Table \ref{table:f:fract:comparison:operators}.
The operators for the boolean type are listed in table 
\ref{table:f:bool:operators}.



\begin{table}[h]
\centering
\begin{tabular}{|c|c|c|}
\hline
\textbf{Operator} & \textbf{Description} & \textbf{Associativity} \\ 
\hline
\verb|+ -| & Unary plus and minus 	& Right-to-left	\\
\verb|* /| & fract multiplication and division & Left-to-right \\ 
\verb|+ -| & fract sum and difference & Left-to-right \\
\hline
\end{tabular}
\caption{F arithmetic operators -- fract types.}
\label{table:f:fract:operators}


\end{table}

\begin{table}[h]
\centering
\begin{tabular}{|c|c|c|}
\hline
\textbf{Operator} & \textbf{Description} & \textbf{Associativity} \\ 
\hline
\verb|<|	& less than	& Right-to-left	\\
\verb|<=|	& less than or equal to	& Left-to-right \\ 
\verb|==|	& equality & Left-to-right \\
\verb|<>|	& disequality & Left-to-right \\
\verb|>|	& greater than & Left-to-right\\
\verb|>=|	& greater than or equal to & Left-to-right \\
\hline
\end{tabular}
\caption{Fract relational operators -- fract types}
\label{table:f:fract:comparison:operators}
\end{table}


\begin{table}[h]
\centering

\begin{tabular}{|c|l|l|l|l|}
\hline
\textbf{Operator} & \textbf{Description} & \textbf{Associativity} \\ 
\hline
\verb|!|	& Logial negation	& Right-to-left	\\
\verb|<->| & logical biimplication (boolean equality) & Left-to-right \\ 
\verb|XOR| & logical xor (boolean disequality) & Left-to-right \\ 
\verb|->|  & logical implication & Left-to-right \\ 
\verb!||!  & logical or & Left-to-right\\ 
\verb|&&|	& logical and & Left-to-right\\ 
\hline
\end{tabular}
\caption{F boolean operators.}
\label{table:f:bool:operators}
\end{table}

\paragraph{Statements}
The \verb|if, if-else, while| statements are very similar to the C one. Each
``block'' is delimited by curly brackets. Each statement is separated by a 
semicolon symbol. You can safely split statements across multiple lines 
provided that its lexeme are valid ones.

\paragraph{Print} To print the variable contents the function \verb|print(ID)|
is provided.

\paragraph{Variable Names} The first character of an identifier starts with
either a letter or an underscore letter. The rest of the string can be freely
build up with letters, numbers and underscores.

\paragraph{Reserved Keywords}
\verb|print, if, else, while, XOR|.



