\section{Parser}
\label{sec:parser}
The main goal of the parser written in bison is to check the
syntax correctness and to produce the Abstract Syntax Tree (AST). 

\subsection{F -- Syntax choice}
We think that a language for students of the middle-school,
who are neophyte programmers, should be statically typed and type safe. 
At the same time it should be easy to use. 
Indeed, these requirements can help the student to learn how a simple 
high-level language works by discriminating between the different types.
Thus we introduce some syntactic sugar, e.g. we 
provide directly the implication operator between boolean expression.

The syntax is very similar to the C one. There is only a slight difference
for the disequality. We preferred the Pascal \verb|<>| over the C \verb|!=|. 

In addition to C we have introduced the following
boolean operators: \verb|<->|, the logical biimplication, 
\verb|->| the implication and \verb|XOR|, the exclusive or.

The syntactic symbols are used \emph{only} in the lexer. In the
rest of the program, i.e. in the parser and in the semantic analyser, only
internal representation of the operators are used. Therefore each grammatic
symbol can be safely replaced without affecting the correctness of FAC.



