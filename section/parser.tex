\section{Parser}
\label{sec:parser}
The main goal of the parser written in bison is to check the
syntax correctness and to produce the Abstract Syntax Tree (AST).

\subsection{F -- Syntax choices}
We think that a language for students of the middle-school,
who are neophyte programmers, should be statically typed and type safe. 
At the same time it should be easy to use. 
Indeed, these requirements can help the student to learn how a simple 
high-level language works by discriminating between the different types.
In particular our minimal type system provides two completely unrelated types.
introduce some syntactic sugar, e.g. we 
provide directly the implication operator between boolean expression.

Thus we discriminated these types by introducing specific boolean
operators (in addition to the one usually provided):
\begin{itemize}
	\item \verb|XOR| - exclusive or;
	\item \verb|<->| - logical biimplication (syntactic sugar);
	\item \verb|->| - logical implication (syntactic sugar);
\end{itemize}

The rest is very similar to the C one. There is only a slight difference
for the arithmetic disequality. We preferred the Pascal \verb|<>| over 
the C \verb|!=|.

The syntactic symbols exist \emph{only} in the lexer. In the
rest of the program, i.e. in the parser and in the semantic analyser, only
internal representation of the operators are used. Therefore each grammar
symbol can be safely replaced without affecting the correctness of FAC.

