\section{Parser}
\label{sec:parser}
The main goal of the parser written in bison is to check the
syntax correctness and to produce the Abstract Syntax Tree (AST).

\subsection{F -- Syntax choices}
We think that a language for students of the middle-school should be
statically typed and type safe; at the same time it should be easy to use.
Indeed, these requirements can help the student to learn how a simple
high-level language works by discriminating between the different types.
In particular, our minimal type system provides two completely unrelated types.


Thus, we discriminated between them by introducing specific type-related
operators. In particular we have defined some boolean operators
(in addition to the ones usually provided\footnote{e.g. in C++, Java}):
\begin{itemize}
	\item \verb|XOR| - exclusive or (syntactic sugar for the boolean disequality);
	\item \verb|<->| - logical biimplication (syntactic sugar);
	\item \verb|->| - logical implication (syntactic sugar);
\end{itemize}

The remaining operators are pretty similar to the ones provided by C.
There is only a slight difference for the arithmetic disequality.
We preferred the Pascal \verb|<>| over the C \verb|!=|.

The syntactic symbols exist \emph{only} in the lexer. In the
rest of the program, i.e. in the parser and in the semantic analyser, we used
only internal representation of the operators. Therefore, each grammar
symbol can be safely replaced without affecting the correctness of FAC.

\subsection{Grammar}
We will not report the whole grammar but \emph{only} its peculiarities.
You can find the complete grammar at \path{FAC/src/parser/parser.y}.

\paragraph{Expressions}
In F you have two types for the variables: fract and boolean.
Originally we distinguished two grammar rules, one for boolean and
one for arithmetic expressions. Unfortunately this distinction
lead to bison conflicts.
Indeed one valid expression built by only one identifier could not be
classified by bison as a boolean or as an airthmetic expression.


So, we decided to simplify the grammar to allow also malformed expressions that
are checked during the type checking phase.

%\begin{verbatim}
%expr,e1, e2 ::= f | b | id | e1 AOP2 e2 | AOP1 expr | !expr | e1 BOP2 e2 | e1 RELOP e2
%\end{verbatim}

\paragraph{Declarations}

In order to avoid possible unitialized variables in the code, the grammar
forces the user to assign a value to a declared variable. This can avoid
typical C undefined behaviours, that can arise when branching are involved, as
demonstrated by the following listing:
\begin{verbatim}
fract f;
while( BEXPR ) {
    // Do some stuff
    f = [1 | 3];
}
//is f initialized or not?
\end{verbatim}
