\section{Parser}
\label{sec:parser}
The main goal of the parser written in bison is to check the
syntax correctness and to produce the Abstract Syntax Tree (AST). 

\subsection{F -- Syntax choice}
We think that a language targeted to girls and boys of the middle-school,
that are neophyte programmers should be statically and strongly typed 
(in order to learn how a programming language works), but at the same
time easy to use. Thus we introduce some syntactic sugar, e.g. we 
provide directly the implication operator between boolean expression.

The syntax is very similar to the C one. There is only a slight difference
for the disequality. We preferred the pascal \verb|<>| over the C \verb|!=|. 

In addition to C we have introduced the following
boolean operators: \verb|<->|, the logical biimplication, 
\verb|->| the implication and \verb|XOR|, the exclusive or.

The syntactic symbols are used \emph{only} in the lexer. In the
rest of the program, i.e. in the parser and in the semantic analyser, only
internal representation of the operators are used. Therefore each grammatic
symbol can be safely replaced without affecting the correctness of FAC.



