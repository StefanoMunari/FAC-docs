\section{Type-Checking}
\label{sec:type-checking}

The type-checker receives as input the AST created by the 
parser. This data structure is convenient to perform
type checking because it gets rid of the syntactic symbols and also because it 
is easily traversable.
In our implementation each operator is wrapped into a category.
For instance the operators \verb|LT, LE, EQ, NEQ, GT, GE| belongs to the 
category \verb|RELOP|. So, we need only a single type system rule for all the
operation of the same category.
This makes the implementation of the type checker more readable.


As already mentioned, in our implementation we distinguish between two types:
boolean and fract. Each time a variable is declared its type is saved into the 
corresponding symbol table entry, thus facilitating the type checking.

So, we implemented a simple type system which exploits the symbol
table as the context. Each time a new variable is declared 
the context is updated with the new variable. This allows us to find whenever
a variable is used before its declaration.
