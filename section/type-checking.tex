\section{Type-Checking}
\label{sec:type-checking}

The type-checker receives as input the AST created by the 
parser. This data structure is really convenient to perform
type checking, because it gets rid of the syntactic symbols.
In our implementation each operator is also grouped into a category.
For instance the operators LT, LE, EQ, NEQ, GT, GE belongs to the category 
RELOP. So, we need only a type system rule for each opeation of the category.
This makes the implementation of the type checker easy to read.


As already mentioned, in our implementation we distinguish between two types:
boolean and fract. Each time a variable is declared its type is saved into the 
corresponding symbol table entry, thus facilitating the type checking.

In other words we implemented a simple type system that exploits the symbol
table as the context. Each time a new variable is declared 
the context is updated with the new variable. This allows us to find whenever
a variable is used before its declaration.
